\chapter{Introduction}
\label{cha:introduction}
\\
\\
The idea of decentralized web is not popular yet, but it will create a huge shift in software industry how companies will be built and one day everyone will be using it. The term “decentralized application” (‘dapp’) doesn’t have a rock solid definition yet, but for our purposes, it means an app that primarily uses data stored on the blockchain. It might otherwise look and feel just like a regular app. Many users are supporting and willing to contribute to decentralize web despite of having a question in mind that why we'll all be using it. Technical experts predicted that the real reason decentralization will win is simply economics. This will affect a huge area of industries from taxis (Uber), to vacation rentals (Airbnb), loans (LendingClub), auctions (eBay), and more. Basically all marketplaces.
\\
\\
Hence currently almost every web and mobile based social, consumer and commercial platforms are centralized and owned by intermediary authorities such as facebook, twitter and instagram etc. Users are creating huge amount of content on these platforms and based on that these authorities are predicting the way future trends looks like, adding a major source of income from advertisements and allowing access to the authorized commercial third parties to the data.This approach of centralization leads to several issues of data protection e.g, lack of data privacy and trust issues. Essentially all the time users are asked to agree on there demanding terms of service license agreement to use content that they submit. Users have access to these centralized applications via centralized marketplaces such as play store, app store and windows store or web based portals, which is again involving an intermediary party. 
\\
\\
To overcome these problems current research aims to decentralize these user centric applications to eradicate single point of control. Distributed ledger technologies, DLT's has been proposed as a solution to these problems of centralization. We believe that a peer-to-peer (P2P) based approach or distributed technologies not only is viable but also highly desirable solution. It can be achieved by blockchain (Distributed Ledger Technology) which ensures that the decentralized nature of blockchain means that it does not rely on a central point of control. A lack of a single authority makes the system fairer and considerably more secure.
\\
\\
Blockchain is a way to store data and do computations on that data through a distributed network of peers, which is governed equally by all participants. With the passage of time, creators of applications publish new versions and after a certain period of time older versions are not supported. Users are forced to adopt the change and what if they don't like the new version and want to continue with older version. In centralize marketplace this seems another major problem from user perspective. In decentralized marketplace once a version is published and users start using it, they can run till whatever time they want, no central authority is involved to force them to update it towards latest one as participants of network has power to decide and it creates an illusion of democracy in the network. In centralized approach all users are using the same version so it's easy to manage while in decentralized network it's near impossible to ensure that all participants are running same implementation with same versions. In a decentralized environment, the communication protocol alone, i.e. the APIs and data formats used, define the type of a service. When agreeing on a specific protocol and version, different implementations of the same service software can hence communicate directly. Anyhow, it needs to be specified what service or API a given user is using and where to get compatible service implementation from. Hence, a decentralized marketplace in which users can browse available APIs and apps is needed.\\
\\
Our idea is to create a decentralized marketplace for decentralize applications using ethereum blockchain technology. This will allow developers to distribute an application without a central provider controlling this whole process. Users can host their data and functionality on their own machines or servers in the cloud. This will allow each user to host his own images, messages, videos and personal contact information, while connecting to the other user's servers and browse through their feeds. The decentralized approach is highly resource scalable in terms that resources is roughly proportional to the number of users in the network. In order to participate in any specific service, users need to install and run compatible application version on his server.  
\\
\\
It is nearly impossible to ensure that all participating users in this decentralized network run the exact same version of the application. Different implementation of same service software can hence communicate directly, given that agreed on specific protocol and versions.
\\
\\
Hence in our decentralized marketplace users can browse available applications and api's. Apart from this, organizations can register themselves as an entity and publish applications in the marketplace which are managed by users associated with the entity. The application implements a set of API's which also need to be published first in the marketplace, so any user can use any application and API. On exploring application published on this marketplace you will also get to know the set of API's implemented by application. This nature of decentralized marketplace or registry service for publishing apps and api's will amend the perspective of the whole existing services in the industry.
