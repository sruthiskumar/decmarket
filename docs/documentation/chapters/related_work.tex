\chapter{Related Work}
\label{cha:relatedwork}

The development of decentralized marketplaces is driven by a desire to establish systems without a central marketplace provider and without trusted intermediaries. Motivations include reducing barriers of entry, reducing fees, increasing resistance to shut-down, and improving privacy. The future seems really exciting for those who believes that decentralization can disrupt the whole world the way things worked in the past using centralized systems. In centralization a single authority runs the whole process of control, for any given activity. In the last couple of years, innovation came with digital platforms and businesses grows without having dedicated assets. For example Uber, Airbnb, Booking.com, E-commerce etc. A typical use case such as social media channels where users are creating content in a huge amount enabling these companies to understand the trends and needs of the market in the future. These centralized systems are much more susceptible to hacking and security attacks. They have a single point of data collection, which represents a significant design weakness. By nature, centralized entities require a loss of privacy. To use a centralized system, you are required to reveal information and this leads to the issues of data privacy and security as well. With the decentralized technology, information doesn’t have to go through one singular point. As a result, it is much more difficult to track. With blockchain, each user has complete transparency about which data is being collected, and how this accessed.
\\
\\
Hence the decentralization is conversely opposite and runs away from the idea of administration and control. All these social applications are hosted by central organizations and users can find them on a registry or directory or marketplace such as google play store, apple app store and microsoft windows store. Once a user downloads any application it will be connected to the central authority and whole data is controlled by one point. Publishers of the applications need to agree on number of terms and conditions governed by these centralized marketplaces. Hence our idea of decentralized marketplace for decentralized applications and api's has been introduced to eradicate this centralization, so that publishers of decentralized applications do not need to worry about any central authority involved in it. Users can download the dapp from this marketplace and can run them on their own server, home device or somewhere in the cloud.
\\
\\
A little work is done in this area such as Tawki, a decentralized social platform for communication is developed at Telekom Innovation Laboratories. Using this platform users will control their own data, which is stored and managed by personal data storages\cite{westerkamp2019tawki}. Each user can communicate with other peers using Tawki API which allows them to send and receive requests from other users personal storage. With decentralization approach this whole systems act as a peer to peer network of users without involvement of third party. Tawki uses ethereum namespace for both the management of user identities and resolving them to the user particular storage location.\cite{westerkamp2019tawki} 
\\
\\
Another partially decentralized marketplace is implemented by SOFIE using ethereum blockchain. The SOFIE project is targeting three industries such as energy, food-chain and mixed reality mobile gaming.\cite{WEBSITE:2} There are different actors in each system and they can offer their services and products among each other. Taking this idea further, a decentralized marketplace is a marketplace that doesn't have a single entity owning or managing it, which in turn enhances its security, resiliency, transparency, and traceability. The decentralized marketplace can be partially decentralized, where for example a group of independent agriculture producers and retailers are managing it, or a fully decentralized where anyone can join and use the marketplace.
\\
\\
Another decentralized marketplace for buyers and sellers is proposed named Desema, a decentralized service marketplace prototype, is a first implementation of this concept that is based on the Ethereum blockchain in combination with IPFS, a peer-to-peer distributed file system. Overall the idea of decentralized marketplace in different domains in not attracted by critical mass but in future it will create potential impact due to its various properties.\cite{Klems2017TrustlessII}
